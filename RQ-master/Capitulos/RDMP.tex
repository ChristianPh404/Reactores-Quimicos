\chapter{Reactor Discontinuo Mezcla perfecta}
\nt{\begin{raggedright}
	en un reactor biologico tenemos que f/v velocidad de dilucion (D)mientras que en otros es v/f , en el primer caso es por que se tiene
	en cuenta los caudales de salida frente los otros son de entrada, tiempos de residencia corto, en el segundo caso es tiempo espacia,
	teniendo en cuenta los caudales de entrada
\end{raggedright}}
\subsection{Isotermo}
\begin{raggedright}
	\textbf{Balance de materia:}
	\begin{equation}
		\frac{dN_i}{dt} = r_i \cdot V
	\end{equation}
	\textbf{Balance de energía:}
	\begin{equation}
		\frac{dT}{dt} = \frac{\sum_{k=1}^{n} \Delta H_k \cdot r_k}{\rho \cdot cp} + \frac{Q(T)}{V \cdot \rho \cdot cp}
	\end{equation}
\textbf{2 posibles cuestiones}
\begin{itemize}
	\item \textbf{Calculo de Volumen:} V
	\item \textbf{Calculo de Tiempo de residencia:} $\tau = \frac{V}{F}$
\end{itemize}
Dentro de los reactivos uno sera el limitante, en la etapa de diseño va referida al reactivo limitante (A)
\textbf{Refiriendonos a la conversion puesto que es el parametro que mas informacion nos da:}
ya que queremos un $\%$ minimo de conversion, se puede calcular como:
\begin{equation}
	X_A = \frac{N_{AR}-N_A}{N_{AR}} \rightarrow N_A = N_{AR} \cdot (1-X_A)
\end{equation}
en la etapa de diseño la masa dara el volumen y el balance entalpico como la temperatura es constante dara el diseño del intercambiador de calor.
por tanto el objetivo a calcular es el tiempo de residencia.
\begin{equation}
	\frac{dN_A}{dt} = -r_A \cdot V
\end{equation}
pars transformar en conversion:
\begin{equation}
	X_A = \frac{N_{AR}-N_A}{N_{AR}} \rightarrow N_A = N_{AR} \cdot (1-X_A)
\end{equation}
\begin{equation}
	\frac{dN_{AR}dX_A}{dt} = r_A \cdot V
\end{equation}
\begin{equation}
	\frac{dX_A}{dt} = \frac{r_A \cdot V}{N_{AR}}
\end{equation}
\begin{equation}
	NAR \int_{XA0}^{XA} \frac{dX_A}{r_A \cdot V} = t_R = NAR \int_{XA0}^{XA} \frac{dXA}{ra V}
\end{equation}
2 casos 
\begin{itemize}
	\item Fase liquida v =cte 
	\begin{equation}
		Tr = CAR \int_{XA0}^{XA} \frac{dxa}{ra}
	\end{equation}
	\item para fase gaseosa 
	\begin{equation}
		Tr = CAR \int_{XA0}^{XA} \frac{dxa}{ra \cdot (1 + eA + ra)}
	\end{equation}
	\end{itemize}
\end{raggedright}
\nt{V  = VR (1 + CA + XA)}
\mlenma{Para componentes que no sean referencia}{CB = CB0 -CA0 * XA \\
en el TR el divisior ra seria k1 * CA que era CA0(1-xA)* lo definido arriba}
\begin{raggedright}
	\textbf{Funcionamiento:}
\nt{Productividad = moles producto deseado /m3/s el tiempo de productividad depende del ciclo no es el de operacion del reactor sino que
incluye el de limpieza acondicionado llenado..}
Si ya esta diseñado y quiero que responda al funcionamiento y no al disño del reactor ci = f(t)
\begin{equation}
	\frac{dCA}{dt} = ra
\end{equation}

\begin{equation}
	\frac{dCB}{dt} = rb
\end{equation}

\begin{equation}
	\frac{dCC}{dt} = rc
\end{equation}
%! Eso para todos los balances ponerlo mas tarde en un multicol...
teniendo V conocido que viene de la etapa de diseño y esta construido el intercambiador de calor y conociendo los parametros Tj T..
La T define las constantes del proceso es decir las K que tambien seran conocidas.
es decir el modelo se completa conociendo r1,r2,r3,ra,rb,rc....rf tambien se puede deducir de forma implicita que simplemente con
los balances a f metiendo las cineticas de r1...r3 aunque recomienda ser explicito para luego poder variar el estudio del sistema.
\subsubsection{Analisis de varibles}
serian 15 ecuaciones las 6 de los componentes las 3 de reacciones y las 6 de los balances, el numero de variables. todas las concentraciones
todas las r y las constantes es decir k1 k2 k3 y k-3 
numero de grados de libertad 4 es decir debe conocerse las 4 constantes cineticas a esa temperatura
\clm{Prueba1}{}{Para el sistema de reacción en serie-paralelo
\begin{gather*}
    2A + B \rightarrow P \\
    P + B \rightarrow Q
\end{gather*}

Propón el modelo que permita calcular la evolución de las concentraciones de todos los componentes y la temperatura cuando en un RDMP a) adiabático, b) no isotermo-no adiabático.
En cada caso realiza el análisis de variables indicando los datos que deben determinarse para la resolución del modelo.}
\qs{RDMP-1}{En un reactor discontinuo mezcla perfecta isotermo, se lleva a cabo la siguiente reacción química:
A $\rightarrow$ B \\
Constante de velocidad = 0.5 h$^{-1}$
El reactor se pone en marcha con 1 mol/L de A. 
\begin{enumerate}[label=\bfseries\tiny\protect\circled{\small\Alph*}]
	\item Obtener la dinámica durante 5 h de reacción, es decir, la evolución temporal de las concentraciones de A y B, así como de la conversión  
\end{enumerate}}

\qs{RDMP-2}{La reacción química
A $\rightarrow$ B

Factor preexponencial = 4.15E5 s$^{-1}$\\
Energía de activación = 11200 cal/mol\\
Entalpía de reacción = -50400  cal/mol

se lleva a cabo en un reactor discontinuo mezcla perfecta adiabático que se pone en marcha con 0.5 mol/L de A a 285 K.
Considerando para la mezcla reaccionante una capacidad calorífica de 0.9 cal/(g*K) y una densidad de 1070 g/L:
\begin{enumerate}[label=\bfseries\tiny\protect\circled{\small\Alph*}]
	\item Obtener la dinámica del proceso
	\item Determinar la temperatura cuando se alcanza el 90$\%$ de conversión
\end{enumerate}}

\qs{RDMP-3a}{La reacción química
A $\rightarrow$ B

Factor preexponencial = 2.2E4 s$^{-1}$\\
Energía de activación = 41570  J/mol\\
Entalpía de reacción = -5E5  J/mol

se lleva a cabo durante 1500 s en un reactor discontinuo mezcla perfecta de 1 m$^3$ que dispone de una camisa de 4 m$^2$ a 283 K 
con un coeficiente global de transmisión de calor de 400 J/(m2·s·K).
Considerando para la mezcla reaccionante una densidad de 980 kg/m$^3$ y una capacidad calorífica de 4200 J/(kg·K):
\begin{enumerate}[label=\bfseries\tiny\protect\circled{\small\Alph*}]
	\item Obtener la dinámica del proceso si el reactor se pone en marcha con 500 mol/m$^3$ de A a 283 K
	\item Localizar el instante de temperatura máxima
\end{enumerate}}

\qs{RDMP-3b}{La reacción química
A $\rightarrow$ B

Factor preexponencial = 2.2E4 s$^{-1}$\\
Energía de activación = 41570  J/mol\\
Entalpía de reacción = -5E5  J/mol

se lleva a cabo durante 1500 s en un reactor discontinuo mezcla perfecta de 1 m$^3$
que dispone de una camisa de 0.1 m$^3$ y 4 m$^2$ capaz de transmitir calor con un coeficiente global de 400 J/(m2·s·K) 
y por la que circulan 0.001 m$^3$/s de agua que entran a 283 K. Considerando para la mezcla reaccionante una densidad de 980 kg/m3 
y una capacidad calorífica de 4200 J/(kg·K):
\begin{enumerate}[label=\bfseries\tiny\protect\circled{\small\Alph*}]
	\item Obtener la dinámica del proceso si el reactor se pone en marcha con 500 mol/m$^3$ de A a 283 K
	\item Determinar el caudal de agua que debe circular por la camisa para que la temperatura del reactor no supere en ningún momento los 330 K
\end{enumerate}}

\qs{RDMP-3c}{La reacción química
A $\rightarrow$ B

Factor preexponencial = 2.2E4 s$^{-1}$\\
Energía de activación = 41570  J/mol\\
Entalpía de reacción = -5E5  J/mol

se lleva a cabo durante 1500 s en un reactor discontinuo mezcla perfecta de 1 m$^3$
que dispone de un serpentín de 13 m de longitud y 0.1 m de diámetro capaz de transmitir calor con un coeficiente 
global de 400 J/(m2·s·K) y por el que circulan 0.001 m$^3$/s de agua que entran a 283 K. Considerando para la mezcla 
reaccionante una densidad de 980 kg/m$^3$ y una capacidad calorífica de 4200 J/(kg·K):
\begin{enumerate}[label=\bfseries\scriptsize\protect\circled{\footnotesize\Alph*}]
	\item Obtener la dinámica del proceso si el reactor se pone en marcha con 500 mol/m$^3$ de A a 283 K
\end{enumerate}}

\qs{RDMP-4}{El equilibrio químico
2A $\rightleftarrows$  B

Factor preexponencial de Arrhenius = 1.4E12 L/(mol·h)\\
Factor preexponencial de Van't Hoff = 6.9E8 L/mol\\
Energía de activación = 105000 J/mol\\
Entalpía de reacción = 63000 J/mol

se lleva a cabo a 420 K en un reactor discontinuo mezcla perfecta con 5 moL/L iniciales de A.
\begin{enumerate}[label=\bfseries\scriptsize\protect\circled{\footnotesize\Alph*}]
	\item Obtener la dinámica del proceso
	\item Determinar cuándo se alcanza el equilibrio, definido por unas derivadas absolutas de las concentraciones inferiores a 1E-5 mol/(L·h)
	\item Localizar el momento en el que las concentraciones de A y B se igualan
\end{enumerate}}

\qs{RDMP-5}{El equilibrio químico

A + B  $\rightleftarrows$  C

Factor preexponencial de Arrhenius = 1.75E8 L/(mol·h)\\
Factor preexponencial de Van't Hoff = 8.25E-22 L/mol\\
Energía de activación = 62350 J/mol\\
Entalpía de reacción = -136400 J/mol

se pone en marcha en un reactor discontinuo mezcla perfecta adiabático con 1 moL/L de A y 2 mol/L de B a 300 K.
 Considerando para la mezcla reaccionante una densidad de 1150 g/L y una capacidad calorífica de 3.8 J/(g·K):
\begin{enumerate}[label=\bfseries\scriptsize\protect\circled{\footnotesize\Alph*}]
	\item Obtener la dinámica del proceso
	\item Determinar cuándo se alcanza el equilibrio, definido por una derivada absoluta de la temperatura inferior a 1E-3 K/h
	\item Estudiar la influencia de la temperatura inicial en la conversión de equilibrio y en el tiempo necesario para alcanzar el 50$\%$ de conversión
\end{enumerate}}

\qs{RDMP-6}{El equilibrio químico

A $\rightleftarrows$ B

Factor preexponencial de Arrhenius directo = 1.94E15 h$^{-1}$\\
Factor preexponencial de Arrhenius inverso = 6.26E19 h$^{-1}$\\
Energía de activación directa = 44500 cal/mol\\
Energía de activación inversa = 59500 cal/mol\\
Temperatura máxima de operación = 650 K

se realiza durante 5 h en un reactor discontinuo mezcla perfecta a partir de 1 mol/L de A.
\begin{enumerate}[label=\bfseries\scriptsize\protect\circled{\footnotesize\Alph*}]
	\item Para una conversión dada, obtener la relación entre temperatura y velocidad de reacción
	\item Determinar la progresión óptima de temperatura
\end{enumerate}}

\qs{RDMP-MULT-1}{Un reactor discontinuo mezcla perfecta de 1 m$^3$ se pone en marcha a 303 K con 1 mol/L del compuesto A, 
transcurriendo las siguientes reacciones químicas en disolución acuosa sin generación apreciable de calor durante 2 h:

A $\rightarrow$ B  , k$_1$ = exp(-1500/T +\hspace{0.5\baselineskip}  6) h$^{-1}$\\
A $\rightarrow$ C  , k$_2$ = exp(-4000/T + 12) h$^{-1}$\\
B $\rightarrow$ D  , k$_3$ = exp(-3000/T + 10) h$^{-1}$

\begin{enumerate}[label=\bfseries\scriptsize\protect\circled{\footnotesize\Alph*}]
	\item Considerando que D es el compuesto deseado, determinar cómo debe operar la camisa 
\end{enumerate}}


\qs{RDMP-MULT-1}{Un reactor discontinuo mezcla perfecta de 1 m$^3$ se pone en marcha a 303 K con 1 mol/L del compuesto A, 
transcurriendo las siguientes reacciones químicas en disolución acuosa sin generación apreciable de calor durante 2 h:

A $\rightarrow$ B  , k$_1$ = exp(-1500/T +\hspace{0.5\baselineskip}  6) h$^{-1}$\\
A $\rightarrow$ C  , k$_2$ = exp(-4000/T + 12) h$^{-1}$\\
B $\rightarrow$ D  , k$_3$ = exp(-3000/T + 10) h$^{-1}$

\begin{enumerate}[label=\bfseries\scriptsize\protect\circled{\footnotesize\Alph*}]
	\item Considerando que D es el compuesto deseado, determinar cómo debe operar la camisa 
\end{enumerate}}

\qs{RDMP-O-2023}{La reacción química exotérmica

A $\rightarrow$ B
k = 0.0231 min$^{-1}$  ,  (T = 280 K)\\
k = 0.0231 min$^{-1}$  ,   (T = 280 K)

Se lleva a cabo en disolución acuosa durante 100 min en un reactor discontinuo mezcla perfecta de 2500 L que dispone de una 
camisa a 280 K capaz de refrigerar a 300 Kj/(min $\cdot$ K). El reactor se pone en marcha con 0.25 mol/L del compuesto A a 280 K.
\begin{enumerate}[label=\bfseries\scriptsize\protect\circled{\footnotesize\Alph*}]
	\item Obtener la dinámica del proceso considerando que el reactor alcanza una temperatura maxima de 290 K
	\item Determinar durante cuánto tiempo la temperatura está comprendida entre 284 y 288 K 
\end{enumerate}}


\qs{RDMP-O-2018}{En un reactor discontinuo mezcla perfecta en fase liquida tiene lugar la reacción quimica exotérmica

A $\rightarrow$ B
Factor preexponencial = 2.5E4 s$^{-1}$\\
Energía de activación = 42000 J/mol\\
Entalpía de reacción = -5E5 J/mol


El reactor, de 5 m$^3$ de volumen, opera durante 1200s. Dispone de una camisa de 1 m$^3$  de volumen y 10 m$^2$ 
de área, a la que entran 0.01 m$^3$/s de agua a 280 K, y que refrigera el reactor con un coeficiente de 500 J/(m$^2$·s·K).

Inicialmente, el reactor contiene 500 mol/m$^3$ de A a 280 K y la camisa se encuentra a la misma temperatura.
\begin{enumerate}[label=\bfseries\scriptsize\protect\circled{\footnotesize\Alph*}]
	\item Calcular la dinámica del proceso
	\item Localizar el 50$\%$ de conversión

	\item Determinar durante cuánto tiempo la temperatura está comprendida entre 300 y 310 K 
\end{enumerate}

Asumir, cuando sea necesario, las propiedades físicas del agua.}

%* __________________________________________________________________________________________________________________
%* -------------------------------------------------Alinear bien las ecuaciones---------------------------------------
\qs{RDMP-O-?}{Las siguientes reacciones químicas se llevan a cabo en un reactor discontinuo mezcla perfecta.

\begin{align*}
    A   &\rightarrow X  & k_1 &= 1 \text{ min}^{-1} \\
    2X + Y &\rightarrow 3X  & k_2 &= 1 \text{ L}^2/(\text{mol}^2 \cdot \text{min}) \\
    B + X &\rightarrow Y + D  & k_3 &= 1 \text{ L}/(\text{mol} \cdot \text{min}) \\
    X &\rightarrow E  & k_4 &= 1 \text{ min}^{-1}
\end{align*}

Las concentraciones de los compuestos A y B (que están en exceso en todo momento) pueden considerarse constantes e iguales a 1.5 y 3 mol/L, respectivamente.

\begin{enumerate}[label=\bfseries\scriptsize\protect\circled{\footnotesize\Alph*}]
    \item Representar gráficamente los balances de materia para los compuestos X e Y.
    \item Obtener los posibles estados estacionarios y determinar su estabilidad.
    \item Representar el campo vectorial.
    \item Si el reactor se pone en marcha sin los compuestos X e Y, determinar durante cuánto tiempo las concentraciones de dichos compuestos son simultáneamente mayores de 2 mol/L.
\end{enumerate}}

\qs{RDMP-O-2022}{El equilibrio químico

A $ \rightleftarrows$ B
Factor preexponencial de Arrhenius = 9.1E4 min$^{-1}$\\
Factor preexponencial de Van't Hoff = 4.21E-5 \\
Energía de activación = 1.12E5 J/mol\\
Entalpía de reacción = -1.46E5 J/mol

Se lleva a cabo un reactor discontinuo mezcla perfecta de 2000L equipado con una camisa a 280 K capaz de refrigerar a 1500 J/(min·K). Para la mezcla reaccionante,
Puede suponerse una densidad de 0.9 Kg/L y una capacidad calorífica de 4500 J/(kg·K).
\begin{enumerate}[label=\bfseries\scriptsize\protect\circled{\footnotesize\Alph*}]
	\item Obtener la dinámica de la reacción hasta alcanzar el 95$\%$ de conversión, considerando que se pone en marcha a 360 K con 1 mol/L del compuesto A
	\item Localizar el punto de inflexión en la evolucion temporal de la temperatura.
\end{enumerate}}

\qs{RDMP-E-2022}{Las siguientes reacciones químicas se llevan a cabo en disolución acuosa durante 2.5h en una reactor discontinuo mezcla perfecta
de 125 cm$^3$ que se pone en marcha a 310 K con 1 mmol/cm$^3$ del compuesto A:

2A $\rightarrow$ B
Factor preexponencial = 6.1E17 cm$^3$/(h·mmol)\\
Energía de activación = 28 cal/mmol \\
Entalpía de reacción = -25 cal/mmol

A $\rightarrow$ C
Factor preexponencial = 5.7E14 h$^{-1}$\\
Energía de activación = 21 cal/mmol\\
Entalpía de reacción = -20 cal/mmol

El reactor dispone de una camisa de 75 cm$^3$ a 300 K con un coeficiente global de transmisión de calor de 10 cal/(cm$^3$·h·K) y cuya temperatura
aumenta inicialmente desde 310 K hasta 312 K.

\begin{enumerate}[label=\bfseries\scriptsize\protect\circled{\footnotesize\Alph*}]
	\item Determinar durante cuánto tiempo la concentración de A es inferior a 0.1 mmol/cm$^3$
	\item Localizar los extremos en la evolución temporal de la temperatura.
\end{enumerate}}

\qs{RDMP-O-?}{El equilibrio químico
A + B$ \rightleftarrows$ C
Factor preexponencial de Arrhenius = 1.8E8 m$^3$/(kmol·h)\\
Factor preexponencial de Van't Hoff = 2.5E-22 m$^3$/kmol\\
Energía de activación = 1.5E4 kcal/kmol\\
Entalpía de reacción = -3.5E4 kcal/kmol

Se lleva a cabo durante 120h en un reactor discontinuo mezcla perfecta de 1 m$^3$ que contiene inicialmente 1 kmol/m$^3$ del compuesto A y 1.5 kmol/m$^3$ del compuesto B a 320 K.
El reactor dispone de una camisa de 5 m$^2$ a 300 K con un coeficiente global de transmision de calor de 1.5 kcal/(kg·K).

Considerando para la mezcla reaccionante una densidad de 1200 kg/m$^3$ y una capacidad calorífica de 0.9 kcal/(kg·K):
\begin{enumerate}[label=\bfseries\scriptsize\protect\circled{\footnotesize\Alph*}]
	\item Determinar las concentraciones intermedias y finales
	\item Determinar durante cuánto tiempo la temperatura del reactor es mayor de 330 K
	\item Optimizar la temperatura de la camisa 
\end{enumerate}}

\nt{en el apartado C se refiere a encontrar la temperatura de la camisa para la que se consigue la mayor conversión}

\qs{RDMP-E-2020}{En un reactor discontinuo mezcla perfecta de 1000 L se llevan a cabo durante 120 min las siguientes reacciones químicas en disolución acuosa:
A $\rightarrow$ B
Factor preexponencial = 6.72 min$^{-1}$\\
Energía de activación = 2980 cal/mol\\
Entalpía de reacción = -1E4 cal/mol

B $\rightarrow$ C
Factor preexponencial = 367 min$^{-1}$\\
Energía de activación = 5960 cal/mol\\
Entalpía de reacción = -2E4 cal/mol

El reactor dispone de una camisa de 100 L capaz de transmitir calor a 1E4 cal/(min·K). y por la que circulan 25 L/min de agua que entran a 280 K.

\begin{enumerate}[label=\bfseries\scriptsize\protect\circled{\footnotesize\Alph*}]
	\item Obtener la dinámica del proceso considerando que inicialmente el reactor contiene 1 mol/l del compuesto A y las temperaturas del reactor
	y camisa se sitúan en 300 K y 280 K, respectivamente
	\item Localizar los cruces de las curvas del grafico de concentraciones
	\item Localizar el momento de máxima transmision de calor.
	\item Determinar el intervalo de valores del caudal de refrigeración que permiten superar una concentración final del compuesto C de 0.85 mol/L manteniendo siempre el reactor
	por debajo de 310 K
\end{enumerate}}

\qs{RDMP-O-2017}{En un reactor discontinuo mezcla perfecta se practica el equilibrio químico
A + B $\rightleftarrows$ P 

con generacion de calor pero sin transmisión. elaborar un script de scilab que permita obtener la dinamica del proceso y localizar una determinada conversión.}

\qs{RDMP-?-?}{La reacción quimica exotérmica\\
A $\rightarrow$ B
Factor preexponencial = 2.5E4 s$^{-1}$\\
k = 1E8*exp(-8000/T) s$^{-1}$\\
H = -2E5 J/mol

Se lleva a cabo en fase acuosa en un reactor discontinuo de 10m$^3$ durante 10000s. El reactor se pone en marcha con una concentración inicial de A
igual a 1000mol/m$^3$ y a 280 K. Se emplea una camisa de 1m$^3$ y 10m$^2$ que esta inicialmente a 280k y a la que se alimentan de 0.01m$^3$/s de agua a 280K. 
El coeficiente global de transmisión de calor es de 400J/(m$^2$·s·K).

\begin{enumerate}[label=\bfseries\tiny\protect\circled{\small\Alph*}]
	\item Aportar scripts salidas en consola, graficas y comentarios para estudiar la dinamica del proceso
\end{enumerate}}

%! ------------------------------------------------------------------------------------------
%! -------------------------------------------RCMP-------------------------------------------
%! ------------------------------------------------------------------------------------------
