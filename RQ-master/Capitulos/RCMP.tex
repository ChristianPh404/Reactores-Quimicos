\chapter{Reactor Continuo Mezcla Perfecta}
\qs{RCMP-E-2023}{El equilibrio químico en disolución acuosa:

A + B  $\rightleftarrows$  C

Factor preexponencial de Arrhenius = 960 m$^3$/(mol·h)\\
Factor preexponencial de Van't Hoff = 1.1E-4 m$^3$/mol\\
Energía de activación = 3.2E4 J/mol\\
Entalpía de reacción = -1.7E5 J/mol

se lleva a cabo en un reactor continuo mezcla perfecta de 1 m$^3$ alimentado por una corriente de 1 m$^3$/h con 1000 mol/m$^3$ del compuesto A 
y 1500 mol/m$^3$ del compuesto B a 300 K. El reactor dispone de una camisa a 290 K capaz de transmitir calor a 9E5 J/(K·h).

\begin{enumerate}[label=\bfseries\scriptsize\protect\circled{\footnotesize\Alph*}]
	\item Empleando un sistema de ecuaciones algebraicas:
	 	\begin{enumerate}[label=\bfseries\tiny\protect\circled{\small\arabic*}]
			\item Obtener el estado estacionario usando la alimentación como solución supuesta
			\item Comprobar la estabilidad del estado estacionario calculado
		\end{enumerate}
	\item Empleando un sistema de ecuaciones diferenciales:
		\begin{enumerate}[label=\bfseries\tiny\protect\circled{\small\arabic*}]
			\item Obtener la dinámica tomando la alimentación como puesta en marcha 
			\item Determinar cuándo la concentración del compuesto C supera a la del compuesto A y la del compuesto B
			\item Determinar cuándo se alcanza la temperatura máxima
		\end{enumerate}
\end{enumerate}
}

\qs{RCMP-O-2020}{Las reacciones químicas\\
A $\rightarrow$ B\\
Factor preexponencial = 4.03E2 min$^{-1}$\\
Energía de activación = 5166 cal/mol\\
Entalpía de reacción = -1000 cal/mol

B $\rightarrow$ C\\
Factor preexponencial = 2.41E7 min$^{-1}$\\
Energía de activación = 12319 cal/mol\\
Entalpía de reacción = -500 cal/mol

B $\rightarrow$ D\\
Factor preexponencial = 1.09E3 min$^{-1}$\\
Energía de activación = 4371 cal/mol\\
Entalpía de reacción = -2000 cal/mol

Se llevan a cabo en disolución acuosa en un reactor continuo mezcla perfecta de 150 L alimentado por una corriente de 5L/min a 300 K con 1 mol/L  del compuesto A.
El reactor dispone de una camisa a 350 K capaz de transmitir calor a 700 cal/(min·K).

\begin{enumerate}[label=\bfseries\scriptsize\protect\circled{\footnotesize\Alph*}]
	\item Creando un primer script 
	 	\begin{enumerate}[label=\bfseries\tiny\protect\circled{\small\arabic*}]
			\item Obtener el estado estacionario del proceso
			\item Optimizar la temperatura de alimentación considerando que D es el compuesto deseado
		\end{enumerate}
	\item Creando un segundo script
		\begin{enumerate}[label=\bfseries\tiny\protect\circled{\small\arabic*}]
			\item Obtener la dinámica para una puesta en marcha en las mismas condiciones que la alimentación hasta alcanzar el estado estacionario 
			definido por una variación de la temperatura inferior a 0.001 K/min
			\item Determinar durante cuánto tiempo predomina el compuesto A
			\item Localizar el momento de máxima concentración del compuesto B
			\item Localizar el momento en el que la concentracion del compuesto C supera a la concentración del compuesto B
			\item Determinar durante cuánto tiempo la concentración del compuesto D esta comprendida entre 0.2 y 0.6 mol/L 
			\item Localizar el momento en el que la temperatura crece a 0.1 K/min 
		\end{enumerate}
\end{enumerate}}

\qs{RCMP-E-2023}{En un reactor continuo mezcla perfecta de 2000 L alimentado por una corriente de 1200 L/min de 2 mol/L del compuesto A  a 300 K se lleva a cabo la reacción química:

A $\rightarrow$ B

Factor preexponencial = 1.5E12 min$^{-1}$\\
Energía de activación = 87.8 Kj/mol\\
Entalpía de reacción = -293 Kj/mol\\
Densidad = 0.9 Kg/L\\
Capacidad calorífica = 3.34 KJ/(Kg·K)

El reactor dispone de una camisa capaz de refrigerar a 2510 Kj/(K·min) que se situa a 370 K durante los primeros 60 min de operación. A continuación, la camisa se sitúa permanentemente a 380 K.


\begin{enumerate}[label=\bfseries\scriptsize\protect\circled{\footnotesize\Alph*}]
	\item Representar gráficamente los balances de materia y energía en estado estacionario y localizar los posibles estados estacionarios.
	\item Para unas condiciones iniciales iguales a la alimentación, obtener la evolución temporal de la concentración de A y de la temperatura y representar la trayectoria en la gráfica anterior.
\end{enumerate}}

\end{raggedright}
