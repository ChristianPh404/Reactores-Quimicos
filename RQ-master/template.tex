\documentclass{report}

\input{preamble} % revisar en preamble \setcounter{tocdepth}{3}
\input{macros}
\input{letterfonts}
\usepackage{graphicx}
\usepackage{chemfig}
\usepackage{multicol}
\usepackage{amsmath}
\title{\Huge{Reactores Quimicos}\\Teoria y ordenador}
\author{\huge{Christian Perez hita}}
\date{}

\begin{document}
\setcounter{tocdepth}{3}
\maketitle
\newpage% or \cleardoublepage
% \pdfbookmark[<level>]{<title>}{<dest>}
\pdfbookmark[section]{\contentsname}{contenidos}

\tableofcontents
\pagebreak

\chapter{Analisis}
\section{Caracterización de reactores}
\dfn{ tipos de Reactores ideales}{se pueden cllasificar en funcion de: \\ $\bullet$ grado de mezcla\\$\bullet$ condiciones de operacion}	
\noindent en funcion del grado de mezcla se clasifican en:
\begin{itemize}
	\item Reactores mezcla perfecta
	\item Reactor de flujo piston
\end{itemize}
\par
\noindent un reactor mezcla perfecta es aquel en el que en todos los puntos del reactor de forma axial y radial la concentracion es la misma,
 en cambio en un reactor de flujo piston la concentracion varia en funcion de la posicion en el reactor.\\

 \noindent  en funcion del intercambio de calor en:
 \begin{itemize}
	\item Reactores isotermicos
	\item Reactores adiabaticos
	\item Reactores no isotermicos-noadiabaticos
\end{itemize}
\noindent en funcion del regimen de operacion:
\begin{itemize}
	\item Reactores continuos
	\item Reactores discontinuos
	\item Reactores semicontinuos
\end{itemize}
\noindent Aunque existen tres posibles configuraciones de reactores según el régimen de operación, es importante recordar que, en el régimen continuo, transcurrido un tiempo 
t, el sistema alcanza un estado estacionario en el cual las propiedades del reactor no varían con el tiempo (ecuaciones algebraicas).\\

\noindent Esto es fundamental para el diseño del reactor. Sin embargo, para un adecuado control del proceso, es necesario realizar un estudio dinámico del reactor, analizando su comportamiento durante la puesta en marcha y ante distintos tipos de estímulos (ecuaciones diferenciales).\\

\noindent De manera evidente, un reactor de flujo pistón es un reactor continuo, ya que no existe otra forma de operación para este tipo de sistema.
\subsection{Reactores de mezcla perfecta}
podemos encontrar 3 tipos de reactores de mezcla perfecta: 
\begin{itemize}
	\item Reactor de tanque agitado continuo
	\item Reactor de tanque agitado discontinuo
	\item Reactor semicontinuo
\end{itemize}
\vspace{2\baselineskip}
\noindent el reactor de tanque agitado continuo (RCMP) es el mas comun, hay tanto una corriente de entrada como una salida\\

\noindent El reactor de tanque agitado discontinuo (RDMP) es un reactor en el que se trabaja por lotes tambien conocido como reactor batch, en este tipo de reactores se carga el reactivo 
y se deja reaccionar hasta que se obtiene el producto deseado.\\

\noindent el reactor semicontinuo (RSMP) puede tener una corriente de entrada o una de salida, aunque lo mas comun es que tenga una corriente de entrada pero no de salida\\

\includegraphics[width=1\textwidth]{RMP.PNG}

\subsection{Reactor de flujo piston}
\begin{raggedright}
En el reactor flujo de piston no hay mezcla a nivel axial, por tanto vamos a tener perfiles de temperatura y concentracion en funcion de la posicion en el reactor(Z).
por ello es necesario trabajar en estado estacionario, con dv al contrario de mezcla perfecta que siempre trabajamos con V, ademas obviamente solo se puede trabajar en continuo.
\end{raggedright}
\begin{center}
	\includegraphics[width=0.8\textwidth]{RFP.png}
\end{center}

\section{Modelizacion de reactores}
\subsection{Modelo de mezcla perfecta}
\begin{raggedright}
	\vspace{1\baselineskip}
	\noindent \textbf{Balance global en continuo:}
	\begin{equation}
		\frac{dM}{dt}=m_o-m_s
	\end{equation}
	si la masa la establecemos en funcion del volumen se puede establecer con el caudal volumetrico y densidad:
	\begin{equation}
		\frac{d(V\rho)}{dt}=F_o\rho_o-F_s\rho_s
	\end{equation}
	en el caso de trabajar en condiciones isomtermas $\rho$ es constante, y en caso de que trabajemos en un rango pequeño de temperatura podemos considerar un valor medio tal que:
	\begin{equation}
		\frac{dV}{dt}=F_o-F_s \xleftrightarrow{\text{V=cte}} \sum F_o=\sum F_s
	\end{equation}
	De forma que el caudal volumetrico de entrada coincide con el caudal volumetrico de salida.
\end{raggedright}
\nt{Esto solo se cumple cuando trabajamos en un reactor ideal ya que se supone mezcla perfecta,un buen sistema de agitacion que cumple las condiciones de homogeniedad y que mantenga un volumen constante.
 en caso contrario  regulando por control de nivel se varia el acceso de corriente de entrada o de salida se mantenga en volumen constante}
 
\vspace{1\baselineskip}
\noindent \textbf{Balance a componentes:}
 \begin{equation}
	 \frac{dN_i}{dt}=ni_o-ni_s +ri \cdot V
 \end{equation}
\begin{flushleft}
\noindent es decir la variacion del numero de moles del componente i respecto al tiempo es igual a la diferencia de moles de entrada y salida mas la velocidad de reacción del componente i por el volumen del reactor (Termino de generacion).\\

 a su vez aunque debemos trabajar en moles puesto que estamos en una reacción quimica, nos interesa mas trabajar en concentraciones ya que nos da mucha mas informacion, y como
 $V\cdot C_i =N_i$ podemos reescribir la ecuacion anterior de la siguiente forma:

\end{flushleft}
 \begin{equation}
	 \frac{d(C_i\cdot V)}{dt}=C_{io}\cdot F_o-C_{is}\cdot F_s +r_i \cdot V
 \end{equation}
\begin{flushleft}

\noindent de donde como se ha mencionado anteriormente, V al ser constante puede salir de la derivada, el caudal volumetrico de entrada es igual al de salida (1.3), ademas como estamos en una mezcla perfecta.\\ 
La concentracion en la entrada y salida del reactor deben de ser identicas ya que para ser una mezcla perfecta en cualquier punto del reactor debe tener las mismas propiedades $\rightarrow C_i=C_{is}$, por tanto se simplifica a la siguiente ecuacion:

\end{flushleft}
 \begin{equation}
	V\frac{dC_i}{dt}=F\cdot (C_{io}-C_{is}) +V \cdot r_i \hspace{0.2cm}\xleftrightarrow{\;\qquad} \hspace{0.2cm}\frac{dC_i}{dt}=\frac{F}{V}\cdot (C_{io}-C_{is}) +r_i
 \end{equation}
 \noindent F/V tambien se define como el tiempo de residencia $\tau$ del reactor, es decir el tiempo que una particula pasa en el reactor.
\newpage
\vspace{1\baselineskip}
\noindent \textbf{Balance de energía:}
\begin{flushleft}
	\noindent el balance, se trata fundamentalmente de un balance entalpico, es decir si estamos en un proceso endotermico dara una disminucion de la temperatura y si es exotermico un aumento de la temperatura.\\
\end{flushleft}
\clm{}{}{	\begin{equation}
	\sum_{i=A}^{P} (M_i \cdot C_{pi}) \cdot \frac{dT}{dt} = \sum_{i=A}^{P} (F_o \cdot C_{io} \cdot H_{io}) - F_s \cdot C_{is} \cdot H_{is} + V \cdot \sum_{k=1}^{n} \Delta H_k \cdot r_k + Q(T)
\end{equation}}

\begin{flushleft}
	\noindent donde la variacion de energia, se puede calcular como la sumatoria para cada componente de su masa por su capacidad calorifica por la variacion de temperatura respecto del tiempo, es decir es el termino de acumulacion.

\vspace{1\baselineskip}
el termino de entalpia de entrada - el de salida de los componentes se trata de $ \sum_{i=A}^{P} (F_o \cdot C_{io} \cdot H_{io}) - F_s \cdot C_{is} \cdot H_{is}$\\
\vspace{1\baselineskip}
mientras que el termino de generación a diferencia del anterior depende de las reacciones quimica, de ahi que sea el indice k frente al i que era de los componentes.
\begin{equation*}
	V \cdot \sum_{k=1}^{n} \Delta H_k \cdot r_k
\end{equation*}
cada reacción tendra su propia entalpia,la entalpia tiene unidad de J|Kj /mol por ello debe ir multiplicada a la velocidad de reacción referida al mismo componente, de forma que queda unidad de energia/(tiempo*V) por ello es necesario multiplicar el volumen \\
\vspace{1\baselineskip}
por ultimo el termino Q(T) es el termino de intercambio de calor, que salvo que trabajemos en condiciones adiabaticas (para el diseño principalmente) siempre ha de estar.

teniendo en cuenta la expresion general, para poder trabajar con ello hay que tener en cuenta que al estar en ideal,mezcla perfecta se tienen las siguientes consideraciones:
\end{flushleft}
\begin{enumerate}[label=\bfseries\tiny\protect\circled{\small\arabic*}]
\item \label{n:1} T = $T_s$
\item \label{n:2} $\rho$ $\approx$ cte ,\hspace{0.1cm} $\rho$ = $\sum_{i=A}^{P} c_i$
\item \label{n:3} $cp_i$ $\approx$ valores medios, cp
\item \label{n:4} V$\cdot$ $\rho$ = M
\item \label{n:5} en sistemas a presion constante, $\Delta H$ = Cp*T y como en la ecuacion es entrada-salida, $\Delta H$ = Cp*(T$_o$-T$_s$)
\end{enumerate}
es decir nos quedaria la siguiente expresion:
\begin{equation}
	V \cdot \rho \cdot cp \cdot \frac{dT}{dt} = F \cdot  \rho \cdot cp \cdot (T_o - T_s) + V \cdot \sum_{k=1}^{n} \Delta H_k \cdot r_k + Q(T)
\end{equation}
simplificando:
\begin{equation}
	\frac{dT}{dt} = \frac{F}{V} \cdot (T_o - T_s) +  \sum_{k=1}^{n} \frac{\Delta H_k \cdot r_k}{\rho \cdot cp} + \frac{Q(T)}{V \cdot \rho \cdot cp}
\end{equation}
\newpage

\subsection{velocidades de reacción}
\vspace{1\baselineskip}
\qs{Calculo de velocidades de reacción}{ 
\schemestart
A + B \arrow{->[1]} C + D \arrow{<->[3]} F
\arrow(@c2--){->[\rotatebox{90}{2}]}[-90] 2E  % Flecha hacia abajo desde C + D
\schemestop}
\sol $r_k$ velocidades de reacción:
\begin{itemize}
	\item $r_1 = k_1 \cdot C_A \cdot C_B$
	\item $r_2 = k_2 \cdot C_C$
	\item $r_3 = k_3 \cdot C_D - k_{-3} \cdot C_F$
\end{itemize}
\sol $r_i$, velocidades de compontentes:
\begin{multicols}{2} % Dividir en 2 columnas
	\begin{itemize}
		\item $r_A = -r_1$
		\item $r_B = -r_1$
		\item $r_C = r_1 - \frac{r_2}{2}$
		\item $r_D = r_1 - r_3$
		\item $r_F = r_3$
		\item $r_E = r_2$
	\end{itemize}
	\end{multicols}

\nt{Las velocidades de reacción, va dirigada a uno de los componentes, normalmente en funcion del producto o del reactivo limitante, Vienen expresadas en terminos de moles/(tiempo*V)}
\subsubsection{Fase Líquida}
la velocidad de reaccion esta en funcion de la constante cinetica (que sera en funcion de la temperatura y de las concentracionees) por tando:
\begin{equation}
	r = f(k,c_i) \rightarrow r=f(T,X_A)
\end{equation}
a su vez, la constante cinetica se obtiene mediante Arrhenius:
\begin{equation}
	k = k_0 \cdot \exp\left(\frac{-E_A}{RT}\right)
\end{equation}
y recordando expresando en funcion de la conversion se puede operar de la siguiente forma:
\begin{equation}
	X_A = \frac{N_{AR}-N_A}{N_{AR}} \rightarrow N_A = N_{AR} \cdot (1-X_A)
\end{equation}
\begin{equation*}
	N_A = f(X_A) \approx C_A = f(X_A)
\end{equation*}
por tanto la concentracion inicial es:
\begin{equation}
	C_{i} = C_{io} - \frac{\alpha_i}{\alpha_A} \cdot (X_A-X_{Ao})
\end{equation}
Siendo $\alpha_i$ el coeficiente estequiometrico del componente i, y $\alpha_A$ el coeficiente estequiometrico del componente que se tomara como reactivo  ademas considerando, en el instante inicial que la conversion es 0:
\begin{equation}
	C_{i} = C_{io} - \frac{\alpha_i}{\alpha_A} \cdot C_{A0} \cdot X_A
\end{equation}
\newpage
\subsubsection{Fase Gaseosa}
En fase gaseosa tenemos 2 posibilidades de trabajo, bien en función de la constante cinetica y concentracion, como en la fase liquida, o bien en funcion de la constante cinetica y Presiones parciales de los componentes.
\begin{equation*}
	r = f(k,P_i) \rightarrow r=f(T,X_A)
\end{equation*}
\begin{equation*}
	n_i = n_{ie} - \frac{\alpha_i}{\alpha_A} \cdot n_{AR} \cdot (X_A-X_{Ae})
\end{equation*}
Recordando que al ser un gas ocupa todo el volumen y segun la ley de los gases ideales:
\begin{equation}
	F = \frac{R \cdot T \cdot n_t}{P}
\end{equation}
donde el numero de moles se relaciona con las presiones de la siguiente forma:
\begin{equation}
	n_t = \sum n_i \rightarrow c_i = \frac{n_i}{F} = \frac{n_i}{n_t}-\frac{P}{R \cdot T} = y_i \cdot \frac{P}{R \cdot T}
\end{equation}
A su vez aplicando la ley de Dalton para las presiones parciales:
\begin{equation}
	P_i = y_i \cdot P = \frac{n_i}{n_t} \cdot P
\end{equation}
es decir sera su fracción molar por la presion total del reactor.
\subsubsection{Calor intercambiado con el exterior}
en un intercambiador de calor, podriamos encontrarno varios casos, bien que solo se intercambie el calor lantente es decir, que no haya variacion de temperatura ya que es solo el cambio de estado, o bien que se intercambie calor sensible, es decir que haya variacion de temperatura.
tambien podria ser una combinacion de ambas.
\begin{itemize}
	\item Intercambio de calor lantente: $T_j \approx cte$ $T_j$ se trata del calor de camisa, de forma que se puede cuantificar como $Q = U\cdot S \cdot (T-T_j)$ es decir se trata del coeficiente global de transferencia de calor por la superficie de intercambio por la diferencia de temperaturas entre el reactor y su camisa.
	\item []el coeficiente de intercambio de calor U, es función de los coeficientes individuales de la transferencia de calor y conductividad termica $U = f(h_i,h_e,k)$
	\item Intercambio de calor sensible: $T_j \neq cte$ en este caso el intercambio de calor se cuantifica como la masa del fluido refrigerante por el calor especifico del mismo por la variacion de temperatura. $Q = m_j \cdot c_{pj} \cdot (T_{js}-T_{je})$
	\item []tambien se podria expresar de forma analoga al intercambio de calor latente pero con la media logaritmica, es decir $Q = U\cdot S \cdot \Delta T_{mlg}$
\end{itemize}
como lo normal es conocer la masa del fluido refrigerante y la temperatura de entrada pero no la de salida se deben correlacionar las ecuaciones anteriores.
\begin{note}
	recordando que la media logaritmica es (la mayor diferencia - la menor diferencia)/ln(mayor/menor) y por logica debe ser menor en la salida que en la entrada.
	\begin{equation*}
		Q = U \cdot S \cdot \frac{(T-T_{je})-(T-T_{js})}{\ln\left(\frac{T-T_{je}}{T-T_{js}}\right)}
	\end{equation*}
\end{note}
\noindent ahora como nos faltaria algun dato para operar con las ecuaciones previas, la forma de correlacionar ambas es la siguiente:
\newpage
\begin{proof}
	\begin{equation*}
		 Q = U\cdot S \cdot (T_{js}-T_{je})
	\end{equation*}\\
	\begin{equation*}
		U \cdot S \cdot \frac{(T{js}-T_{je})}{\ln\left(\frac{T-T_{je}}{T-T_{js}}\right)}
   \end{equation*}
\vspace{0.1cm}
	\begin{equation*}
		m_j \cdot c_{pj} \cdot (T_{js}-T_{je}) = U \cdot S \cdot \frac{(T-T_{je})-(T-T_{js})}{\ln\left(\frac{T-T_{je}}{T-T_{js}}\right)}
	\end{equation*}
	\begin{equation*}
		\ln \left(\frac{T-T_{je}}{T-T_{js}}\right) = \frac{U \cdot S }{m_j \cdot c_{pj}} 
	\end{equation*}\\

	\noindent donde simplemente despejando  de esta ultima ecuación:
	\begin{equation*}
		\frac{T-T_{je}}{T-T_{js}} = \exp\left(\frac{U \cdot S }{m_j \cdot c_{pj}}\right)
	\end{equation*}
	\begin{equation*}
		T - T_{je} = (T-T_{js}) \cdot \exp\left(\frac{U \cdot S }{m_j \cdot c_{pj}}\right) 
	\end{equation*}\\
	al pasar el exponente al otro lado de la ecuacion:\\
	\begin{equation*}
		T-T_{js} = (T - T_{je}) \cdot \exp\left(- \hspace{0.5em}\frac{U \cdot S }{m_j \cdot c_{pj}}\right) 
	\end{equation*}
	\begin{equation*}
		T_{js} = T- (T - T_{je}) \cdot \exp\left(- \hspace{0.5em}\frac{U \cdot S }{m_j \cdot c_{pj}}\right) 
	\end{equation*}
	\noindent y reintroduciendo en la primera expresion queda tal que:
	\begin{equation*}
		Q = m_j \cdot c_{pj} \cdot \left(T_{js} - \left[ T - (T - T_{je}) \cdot \exp\left(- \hspace{0.5em}\frac{U \cdot S }{m_j \cdot c_{pj}}\right) \right] \right)
	\end{equation*}
	\begin{equation*}
		Q = m_j \cdot c_{pj} \cdot (T - T_{je}) \cdot \left[1 - \exp\left(-\hspace{0.5em} \frac{U \cdot S}{m_j \cdot c_{pj}}\right)\right]
	\end{equation*}
\end{proof}
\newpage
\chapter{Reactor Discontinuo Mezcla perfecta}
\nt{\begin{raggedright}
	en un reactor biologico tenemos que f/v velocidad de dilucion (D)mientras que en otros es v/f , en el primer caso es por que se tiene
	en cuenta los caudales de salida frente los otros son de entrada, tiempos de residencia corto, en el segundo caso es tiempo espacia,
	teniendo en cuenta los caudales de entrada
\end{raggedright}}
\subsection{Isotermo}
\begin{raggedright}
	\textbf{Balance de materia:}
	\begin{equation}
		\frac{dN_i}{dt} = r_i \cdot V
	\end{equation}
	\textbf{Balance de energía:}
	\begin{equation}
		\frac{dT}{dt} = \frac{\sum_{k=1}^{n} \Delta H_k \cdot r_k}{\rho \cdot cp} + \frac{Q(T)}{V \cdot \rho \cdot cp}
	\end{equation}
\textbf{2 posibles cuestiones}
\begin{itemize}
	\item \textbf{Calculo de Volumen:} V
	\item \textbf{Calculo de Tiempo de residencia:} $\tau = \frac{V}{F}$
\end{itemize}
Dentro de los reactivos uno sera el limitante, en la etapa de diseño va referida al reactivo limitante (A)
\textbf{Refiriendonos a la conversion puesto que es el parametro que mas informacion nos da:}
ya que queremos un $\%$ minimo de conversion, se puede calcular como:
\begin{equation}
	X_A = \frac{N_{AR}-N_A}{N_{AR}} \rightarrow N_A = N_{AR} \cdot (1-X_A)
\end{equation}
en la etapa de diseño la masa dara el volumen y el balance entalpico como la temperatura es constante dara el diseño del intercambiador de calor.
por tanto el objetivo a calcular es el tiempo de residencia.
\begin{equation}
	\frac{dN_A}{dt} = -r_A \cdot V
\end{equation}
pars transformar en conversion:
\begin{equation}
	X_A = \frac{N_{AR}-N_A}{N_{AR}} \rightarrow N_A = N_{AR} \cdot (1-X_A)
\end{equation}
\begin{equation}
	\frac{dN_{AR}dX_A}{dt} = r_A \cdot V
\end{equation}
\begin{equation}
	\frac{dX_A}{dt} = \frac{r_A \cdot V}{N_{AR}}
\end{equation}
\begin{equation}
	NAR \int_{XA0}^{XA} \frac{dX_A}{r_A \cdot V} = t_R = NAR \int_{XA0}^{XA} \frac{dXA}{ra V}
\end{equation}
2 casos 
\begin{itemize}
	\item Fase liquida v =cte 
	\begin{equation}
		Tr = CAR \int_{XA0}^{XA} \frac{dxa}{ra}
	\end{equation}
	\item para fase gaseosa 
	\begin{equation}
		Tr = CAR \int_{XA0}^{XA} \frac{dxa}{ra \cdot (1 + eA + ra)}
	\end{equation}
	\end{itemize}
\end{raggedright}
\nt{V  = VR (1 + CA + XA)}
\mlenma{Para componentes que no sean referencia}{CB = CB0 -CA0 * XA \\
en el TR el divisior ra seria k1 * CA que era CA0(1-xA)* lo definido arriba}
\begin{raggedright}
	\textbf{Funcionamiento:}
\nt{Productividad = moles producto deseado /m3/s el tiempo de productividad depende del ciclo no es el de operacion del reactor sino que
incluye el de limpieza acondicionado llenado..}
Si ya esta diseñado y quiero que responda al funcionamiento y no al disño del reactor ci = f(t)
\begin{equation}
	\frac{dCA}{dt} = ra
\end{equation}

\begin{equation}
	\frac{dCB}{dt} = rb
\end{equation}

\begin{equation}
	\frac{dCC}{dt} = rc
\end{equation}
%! Eso para todos los balances ponerlo mas tarde en un multicol...
teniendo V conocido que viene de la etapa de diseño y esta construido el intercambiador de calor y conociendo los parametros Tj T..
La T define las constantes del proceso es decir las K que tambien seran conocidas.
es decir el modelo se completa conociendo r1,r2,r3,ra,rb,rc....rf tambien se puede deducir de forma implicita que simplemente con
los balances a f metiendo las cineticas de r1...r3 aunque recomienda ser explicito para luego poder variar el estudio del sistema.
\subsubsection{Analisis de varibles}
serian 15 ecuaciones las 6 de los componentes las 3 de reacciones y las 6 de los balances, el numero de variables. todas las concentraciones
todas las r y las constantes es decir k1 k2 k3 y k-3 
numero de grados de libertad 4 es decir debe conocerse las 4 constantes cineticas a esa temperatura
\clm{Prueba1}{}{Para el sistema de reacción en serie-paralelo\\
\begin{gather*}
    2A + B \rightarrow P \\
    P + B \rightarrow Q
\end{gather*}

Propón el modelo que permita calcular la evolución de las concentraciones de todos los componentes y la temperatura cuando en un RDMP a) adiabático, b) no isotermo-no adiabático.
En cada caso realiza el análisis de variables indicando los datos que deben determinarse para la resolución del modelo.}
\qs{RDMP-1}{En un reactor discontinuo mezcla perfecta isotermo, se lleva a cabo la siguiente reacción química:\\
A $\rightarrow$ B \\
Constante de velocidad = 0.5 h$^{-1}$\\
El reactor se pone en marcha con 1 mol/L de A. 
\begin{enumerate}[label=\bfseries\tiny\protect\circled{\small\Alph*}]
	\item Obtener la dinámica durante 5 h de reacción, es decir, la evolución temporal de las concentraciones de A y B, así como de la conversión  
\end{enumerate}}

\qs{RDMP-2}{La reacción química\\
A $\rightarrow$ B \\

Factor preexponencial = 4.15E5 s$^{-1}$\\
Energía de activación = 11200 cal/mol\\
Entalpía de reacción = -50400  cal/mol\\

se lleva a cabo en un reactor discontinuo mezcla perfecta adiabático que se pone en marcha con 0.5 mol/L de A a 285 K.
Considerando para la mezcla reaccionante una capacidad calorífica de 0.9 cal/(g*K) y una densidad de 1070 g/L:\\
\begin{enumerate}[label=\bfseries\tiny\protect\circled{\small\Alph*}]
	\item Obtener la dinámica del proceso
	\item Determinar la temperatura cuando se alcanza el 90$\%$ de conversión
\end{enumerate}}

\qs{RDMP-3a}{La reacción química\\
A $\rightarrow$ B \\

Factor preexponencial = 2.2E4 s$^{-1}$\\
Energía de activación = 41570  J/mol\\
Entalpía de reacción = -5E5  J/mol\\

se lleva a cabo durante 1500 s en un reactor discontinuo mezcla perfecta de 1 m$^3$ que dispone de una camisa de 4 m$^2$ a 283 K 
con un coeficiente global de transmisión de calor de 400 J/(m2·s·K).
Considerando para la mezcla reaccionante una densidad de 980 kg/m$^3$ y una capacidad calorífica de 4200 J/(kg·K):\\
\begin{enumerate}[label=\bfseries\tiny\protect\circled{\small\Alph*}]
	\item Obtener la dinámica del proceso si el reactor se pone en marcha con 500 mol/m$^3$ de A a 283 K
	\item Localizar el instante de temperatura máxima
\end{enumerate}}

\qs{RDMP-3b}{La reacción química\\
A $\rightarrow$ B \\

Factor preexponencial = 2.2E4 s$^{-1}$\\
Energía de activación = 41570  J/mol\\
Entalpía de reacción = -5E5  J/mol\\

se lleva a cabo durante 1500 s en un reactor discontinuo mezcla perfecta de 1 m$^3$
que dispone de una camisa de 0.1 m$^3$ y 4 m$^2$ capaz de transmitir calor con un coeficiente global de 400 J/(m2·s·K) 
y por la que circulan 0.001 m$^3$/s de agua que entran a 283 K. Considerando para la mezcla reaccionante una densidad de 980 kg/m3 
y una capacidad calorífica de 4200 J/(kg·K):\\
\begin{enumerate}[label=\bfseries\tiny\protect\circled{\small\Alph*}]
	\item Obtener la dinámica del proceso si el reactor se pone en marcha con 500 mol/m$^3$ de A a 283 K
	\item Determinar el caudal de agua que debe circular por la camisa para que la temperatura del reactor no supere en ningún momento los 330 K
\end{enumerate}}

\qs{RDMP-3c}{La reacción química\\
A $\rightarrow$ B \\

Factor preexponencial = 2.2E4 s$^{-1}$\\
Energía de activación = 41570  J/mol\\
Entalpía de reacción = -5E5  J/mol\\

se lleva a cabo durante 1500 s en un reactor discontinuo mezcla perfecta de 1 m$^3$
que dispone de un serpentín de 13 m de longitud y 0.1 m de diámetro capaz de transmitir calor con un coeficiente 
global de 400 J/(m2·s·K) y por el que circulan 0.001 m$^3$/s de agua que entran a 283 K. Considerando para la mezcla 
reaccionante una densidad de 980 kg/m$^3$ y una capacidad calorífica de 4200 J/(kg·K):\\
\begin{enumerate}[label=\bfseries\scriptsize\protect\circled{\footnotesize\Alph*}]
	\item Obtener la dinámica del proceso si el reactor se pone en marcha con 500 mol/m$^3$ de A a 283 K
\end{enumerate}}

\qs{RDMP-4}{El equilibrio químico\\
2A $\rightleftarrows$  B \\

Factor preexponencial de Arrhenius = 1.4E12 L/(mol·h)\\
Factor preexponencial de Van't Hoff = 6.9E8 L/mol\\
Energía de activación = 105000 J/mol\\
Entalpía de reacción = 63000 J/mol\\

se lleva a cabo a 420 K en un reactor discontinuo mezcla perfecta con 5 moL/L iniciales de A.\\
\begin{enumerate}[label=\bfseries\scriptsize\protect\circled{\footnotesize\Alph*}]
	\item Obtener la dinámica del proceso
	\item Determinar cuándo se alcanza el equilibrio, definido por unas derivadas absolutas de las concentraciones inferiores a 1E-5 mol/(L·h)
	\item Localizar el momento en el que las concentraciones de A y B se igualan
\end{enumerate}}

\qs{RDMP-5}{El equilibrio químico\\

A + B  $\rightleftarrows$  C \\

Factor preexponencial de Arrhenius = 1.75E8 L/(mol·h)\\
Factor preexponencial de Van't Hoff = 8.25E-22 L/mol\\
Energía de activación = 62350 J/mol\\
Entalpía de reacción = -136400 J/mol\\

se pone en marcha en un reactor discontinuo mezcla perfecta adiabático con 1 moL/L de A y 2 mol/L de B a 300 K.
 Considerando para la mezcla reaccionante una densidad de 1150 g/L y una capacidad calorífica de 3.8 J/(g·K):
\begin{enumerate}[label=\bfseries\scriptsize\protect\circled{\footnotesize\Alph*}]
	\item Obtener la dinámica del proceso
	\item Determinar cuándo se alcanza el equilibrio, definido por una derivada absoluta de la temperatura inferior a 1E-3 K/h
	\item Estudiar la influencia de la temperatura inicial en la conversión de equilibrio y en el tiempo necesario para alcanzar el 50$\%$ de conversión
\end{enumerate}}

\qs{RDMP-6}{El equilibrio químico\\

A $\rightleftarrows$ B \\

Factor preexponencial de Arrhenius directo = 1.94E15 h$^{-1}$\\
Factor preexponencial de Arrhenius inverso = 6.26E19 h$^{-1}$\\
Energía de activación directa = 44500 cal/mol\\
Energía de activación inversa = 59500 cal/mol\\
Temperatura máxima de operación = 650 K\\

se realiza durante 5 h en un reactor discontinuo mezcla perfecta a partir de 1 mol/L de A.
\begin{enumerate}[label=\bfseries\scriptsize\protect\circled{\footnotesize\Alph*}]
	\item Para una conversión dada, obtener la relación entre temperatura y velocidad de reacción
	\item Determinar la progresión óptima de temperatura
\end{enumerate}}

\qs{RDMP-MULT-1}{Un reactor discontinuo mezcla perfecta de 1 m$^3$ se pone en marcha a 303 K con 1 mol/L del compuesto A, 
transcurriendo las siguientes reacciones químicas en disolución acuosa sin generación apreciable de calor durante 2 h:\\

A $\rightarrow$ B  , k$_1$ = exp(-1500/T +\hspace{0.5\baselineskip}  6) h$^{-1}$\\
A $\rightarrow$ C  , k$_2$ = exp(-4000/T + 12) h$^{-1}$\\
B $\rightarrow$ D  , k$_3$ = exp(-3000/T + 10) h$^{-1}$\\

\begin{enumerate}[label=\bfseries\scriptsize\protect\circled{\footnotesize\Alph*}]
	\item Considerando que D es el compuesto deseado, determinar cómo debe operar la camisa 
\end{enumerate}}


\qs{RDMP-MULT-1}{Un reactor discontinuo mezcla perfecta de 1 m$^3$ se pone en marcha a 303 K con 1 mol/L del compuesto A, 
transcurriendo las siguientes reacciones químicas en disolución acuosa sin generación apreciable de calor durante 2 h:\\

A $\rightarrow$ B  , k$_1$ = exp(-1500/T +\hspace{0.5\baselineskip}  6) h$^{-1}$\\
A $\rightarrow$ C  , k$_2$ = exp(-4000/T + 12) h$^{-1}$\\
B $\rightarrow$ D  , k$_3$ = exp(-3000/T + 10) h$^{-1}$\\

\begin{enumerate}[label=\bfseries\scriptsize\protect\circled{\footnotesize\Alph*}]
	\item Considerando que D es el compuesto deseado, determinar cómo debe operar la camisa 
\end{enumerate}}

\qs{RDMP-O-2023}{La reacción química exotérmica\\

A $\rightarrow$ B \\
k = 0.0231 min$^{-1}$  ,  (T = 280 K)\\
k = 0.0231 min$^{-1}$  ,   (T = 280 K)\\

Se lleva a cabo en disolución acuosa durante 100 min en un reactor discontinuo mezcla perfecta de 2500 L que dispone de una 
camisa a 280 K capaz de refrigerar a 300 Kj/(min $\cdot$ K). El reactor se pone en marcha con 0.25 mol/L del compuesto A a 280 K.\\
\begin{enumerate}[label=\bfseries\scriptsize\protect\circled{\footnotesize\Alph*}]
	\item Obtener la dinámica del proceso considerando que el reactor alcanza una temperatura maxima de 290 K
	\item Determinar durante cuánto tiempo la temperatura está comprendida entre 284 y 288 K 
\end{enumerate}}


\qs{RDMP-O-2018}{En un reactor discontinuo mezcla perfecta en fase liquida tiene lugar la reacción quimica exotérmica\\

A $\rightarrow$ B \\
Factor preexponencial = 2.5E4 s$^{-1}$\\
Energía de activación = 42000 J/mol\\
Entalpía de reacción = -5E5 J/mol\\


El reactor, de 5 m$^3$ de volumen, opera durante 1200s. Dispone de una camisa de 1 m$^3$  de volumen y 10 m$^2$ 
de área, a la que entran 0.01 m$^3$/s de agua a 280 K, y que refrigera el reactor con un coeficiente de 500 J/(m$^2$·s·K).\\

Inicialmente, el reactor contiene 500 mol/m$^3$ de A a 280 K y la camisa se encuentra a la misma temperatura.\\
\begin{enumerate}[label=\bfseries\scriptsize\protect\circled{\footnotesize\Alph*}]
	\item Calcular la dinámica del proceso
	\item Localizar el 50$\%$ de conversión

	\item Determinar durante cuánto tiempo la temperatura está comprendida entre 300 y 310 K 
\end{enumerate}

Asumir, cuando sea necesario, las propiedades físicas del agua.}

%* __________________________________________________________________________________________________________________
%* -------------------------------------------------Alinear bien las ecuaciones---------------------------------------
\qs{RDMP-O-?}{Las siguientes reacciones químicas se llevan a cabo en un reactor discontinuo mezcla perfecta.

\begin{align*}
    A   &\rightarrow X  & k_1 &= 1 \text{ min}^{-1} \\
    2X + Y &\rightarrow 3X  & k_2 &= 1 \text{ L}^2/(\text{mol}^2 \cdot \text{min}) \\
    B + X &\rightarrow Y + D  & k_3 &= 1 \text{ L}/(\text{mol} \cdot \text{min}) \\
    X &\rightarrow E  & k_4 &= 1 \text{ min}^{-1}
\end{align*}

Las concentraciones de los compuestos A y B (que están en exceso en todo momento) pueden considerarse constantes e iguales a 1.5 y 3 mol/L, respectivamente.

\begin{enumerate}[label=\bfseries\scriptsize\protect\circled{\footnotesize\Alph*}]
    \item Representar gráficamente los balances de materia para los compuestos X e Y.
    \item Obtener los posibles estados estacionarios y determinar su estabilidad.
    \item Representar el campo vectorial.
    \item Si el reactor se pone en marcha sin los compuestos X e Y, determinar durante cuánto tiempo las concentraciones de dichos compuestos son simultáneamente mayores de 2 mol/L.
\end{enumerate}}

\qs{RDMP-O-2022}{El equilibrio químico\\

A $ \rightleftarrows$ B \\
Factor preexponencial de Arrhenius = 9.1E4 min$^{-1}$\\
Factor preexponencial de Van't Hoff = 4.21E-5 \\
Energía de activación = 1.12E5 J/mol\\
Entalpía de reacción = -1.46E5 J/mol\\

Se lleva a cabo un reactor discontinuo mezcla perfecta de 2000L equipado con una camisa a 280 K capaz de refrigerar a 1500 J/(min·K). Para la mezcla reaccionante,
Puede suponerse una densidad de 0.9 Kg/L y una capacidad calorífica de 4500 J/(kg·K).\\
\begin{enumerate}[label=\bfseries\scriptsize\protect\circled{\footnotesize\Alph*}]
	\item Obtener la dinámica de la reacción hasta alcanzar el 95$\%$ de conversión, considerando que se pone en marcha a 360 K con 1 mol/L del compuesto A
	\item Localizar el punto de inflexión en la evolucion temporal de la temperatura.
\end{enumerate}}

\qs{RDMP-E-2022}{Las siguientes reacciones químicas se llevan a cabo en disolución acuosa durante 2.5h en una reactor discontinuo mezcla perfecta
de 125 cm$^3$ que se pone en marcha a 310 K con 1 mmol/cm$^3$ del compuesto A:\\

2A $\rightarrow$ B \\
Factor preexponencial = 6.1E17 cm$^3$/(h·mmol)\\
Energía de activación = 28 cal/mmol \\
Entalpía de reacción = -25 cal/mmol\\

A $\rightarrow$ C \\
Factor preexponencial = 5.7E14 h$^{-1}$\\
Energía de activación = 21 cal/mmol\\
Entalpía de reacción = -20 cal/mmol\\

El reactor dispone de una camisa de 75 cm$^3$ a 300 K con un coeficiente global de transmisión de calor de 10 cal/(cm$^3$·h·K) y cuya temperatura
aumenta inicialmente desde 310 K hasta 312 K.

\begin{enumerate}[label=\bfseries\scriptsize\protect\circled{\footnotesize\Alph*}]
	\item Determinar durante cuánto tiempo la concentración de A es inferior a 0.1 mmol/cm$^3$
	\item Localizar los extremos en la evolución temporal de la temperatura.
\end{enumerate}}

\qs{RDMP-O-?}{El equilibrio químico\\
A + B$ \rightleftarrows$ C \\
Factor preexponencial de Arrhenius = 1.8E8 m$^3$/(kmol·h)\\
Factor preexponencial de Van't Hoff = 2.5E-22 m$^3$/kmol\\
Energía de activación = 1.5E4 kcal/kmol\\
Entalpía de reacción = -3.5E4 kcal/kmol\\

Se lleva a cabo durante 120h en un reactor discontinuo mezcla perfecta de 1 m$^3$ que contiene inicialmente 1 kmol/m$^3$ del compuesto A y 1.5 kmol/m$^3$ del compuesto B a 320 K.
El reactor dispone de una camisa de 5 m$^2$ a 300 K con un coeficiente global de transmision de calor de 1.5 kcal/(kg·K).\\

Considerando para la mezcla reaccionante una densidad de 1200 kg/m$^3$ y una capacidad calorífica de 0.9 kcal/(kg·K):\\
\begin{enumerate}[label=\bfseries\scriptsize\protect\circled{\footnotesize\Alph*}]
	\item Determinar las concentraciones intermedias y finales
	\item Determinar durante cuánto tiempo la temperatura del reactor es mayor de 330 K
	\item Optimizar la temperatura de la camisa 
\end{enumerate}}

\nt{en el apartado C se refiere a encontrar la temperatura de la camisa para la que se consigue la mayor conversión}

\qs{RDMP-E-2020}{En un reactor discontinuo mezcla perfecta de 1000 L se llevan a cabo durante 120 min las siguientes reacciones químicas en disolución acuosa:\\
A $\rightarrow$ B \\
Factor preexponencial = 6.72 min$^{-1}$\\
Energía de activación = 2980 cal/mol\\
Entalpía de reacción = -1E4 cal/mol\\

B $\rightarrow$ C \\
Factor preexponencial = 367 min$^{-1}$\\
Energía de activación = 5960 cal/mol\\
Entalpía de reacción = -2E4 cal/mol\\

El reactor dispone de una camisa de 100 L capaz de transmitir calor a 1E4 cal/(min·K). y por la que circulan 25 L/min de agua que entran a 280 K.\\

\begin{enumerate}[label=\bfseries\scriptsize\protect\circled{\footnotesize\Alph*}]
	\item Obtener la dinámica del proceso considerando que inicialmente el reactor contiene 1 mol/l del compuesto A y las temperaturas del reactor
	y camisa se sitúan en 300 K y 280 K, respectivamente
	\item Localizar los cruces de las curvas del grafico de concentraciones
	\item Localizar el momento de máxima transmision de calor.
	\item Determinar el intervalo de valores del caudal de refrigeración que permiten superar una concentración final del compuesto C de 0.85 mol/L manteniendo siempre el reactor
	por debajo de 310 K
\end{enumerate}}

\qs{RDMP-O-2017}{En un reactor discontinuo mezcla perfecta se practica el equilibrio químico\\
A + B $\rightleftarrows$ P \\ 

con generacion de calor pero sin transmisión. elaborar un script de scilab que permita obtener la dinamica del proceso y localizar una determinada conversión.}

\qs{RDMP-?-?}{La reacción quimica exotérmica\\
A $\rightarrow$ B \\
Factor preexponencial = 2.5E4 s$^{-1}$\\
k = 1E8*exp(-8000/T) s$^{-1}$\\
H = -2E5 J/mol\\

Se lleva a cabo en fase acuosa en un reactor discontinuo de 10m$^3$ durante 10000s. El reactor se pone en marcha con una concentración inicial de A
igual a 1000mol/m$^3$ y a 280 K. Se emplea una camisa de 1m$^3$ y 10m$^2$ que esta inicialmente a 280k y a la que se alimentan de 0.01m$^3$/s de agua a 280K. 
El coeficiente global de transmisión de calor es de 400J/(m$^2$·s·K).

\begin{enumerate}[label=\bfseries\tiny\protect\circled{\small\Alph*}]
	\item Aportar scripts salidas en consola, graficas y comentarios para estudiar la dinamica del proceso
\end{enumerate}}

%! ------------------------------------------------------------------------------------------
%! -------------------------------------------RCMP-------------------------------------------
%! ------------------------------------------------------------------------------------------
\chapter{Reactor Continuo Mezcla Perfecta}
\qs{RCMP-E-2023}{El equilibrio químico en disolución acuosa:\\

A + B  $\rightleftarrows$  C \\

Factor preexponencial de Arrhenius = 960 m$^3$/(mol·h)\\
Factor preexponencial de Van't Hoff = 1.1E-4 m$^3$/mol\\
Energía de activación = 3.2E4 J/mol\\
Entalpía de reacción = -1.7E5 J/mol\\

se lleva a cabo en un reactor continuo mezcla perfecta de 1 m$^3$ alimentado por una corriente de 1 m$^3$/h con 1000 mol/m$^3$ del compuesto A 
y 1500 mol/m$^3$ del compuesto B a 300 K. El reactor dispone de una camisa a 290 K capaz de transmitir calor a 9E5 J/(K·h).

\begin{enumerate}[label=\bfseries\scriptsize\protect\circled{\footnotesize\Alph*}]
	\item Empleando un sistema de ecuaciones algebraicas:
	 	\begin{enumerate}[label=\bfseries\tiny\protect\circled{\small\arabic*}]
			\item Obtener el estado estacionario usando la alimentación como solución supuesta
			\item Comprobar la estabilidad del estado estacionario calculado
		\end{enumerate}
	\item Empleando un sistema de ecuaciones diferenciales:
		\begin{enumerate}[label=\bfseries\tiny\protect\circled{\small\arabic*}]
			\item Obtener la dinámica tomando la alimentación como puesta en marcha 
			\item Determinar cuándo la concentración del compuesto C supera a la del compuesto A y la del compuesto B
			\item Determinar cuándo se alcanza la temperatura máxima
		\end{enumerate}
\end{enumerate}
}

\qs{RCMP-O-2020}{Las reacciones químicas\\
A $\rightarrow$ B\\
Factor preexponencial = 4.03E2 min$^{-1}$\\
Energía de activación = 5166 cal/mol\\
Entalpía de reacción = -1000 cal/mol\\

B $\rightarrow$ C\\
Factor preexponencial = 2.41E7 min$^{-1}$\\
Energía de activación = 12319 cal/mol\\
Entalpía de reacción = -500 cal/mol\\

B $\rightarrow$ D\\
Factor preexponencial = 1.09E3 min$^{-1}$\\
Energía de activación = 4371 cal/mol\\
Entalpía de reacción = -2000 cal/mol\\

Se llevan a cabo en disolución acuosa en un reactor continuo mezcla perfecta de 150 L alimentado por una corriente de 5L/min a 300 K con 1 mol/L  del compuesto A.
El reactor dispone de una camisa a 350 K capaz de transmitir calor a 700 cal/(min·K).

\begin{enumerate}[label=\bfseries\scriptsize\protect\circled{\footnotesize\Alph*}]
	\item Creando un primer script 
	 	\begin{enumerate}[label=\bfseries\tiny\protect\circled{\small\arabic*}]
			\item Obtener el estado estacionario del proceso
			\item Optimizar la temperatura de alimentación considerando que D es el compuesto deseado
		\end{enumerate}
	\item Creando un segundo script
		\begin{enumerate}[label=\bfseries\tiny\protect\circled{\small\arabic*}]
			\item Obtener la dinámica para una puesta en marcha en las mismas condiciones que la alimentación hasta alcanzar el estado estacionario 
			definido por una variación de la temperatura inferior a 0.001 K/min
			\item Determinar durante cuánto tiempo predomina el compuesto A
			\item Localizar el momento de máxima concentración del compuesto B
			\item Localizar el momento en el que la concentracion del compuesto C supera a la concentración del compuesto B
			\item Determinar durante cuánto tiempo la concentración del compuesto D esta comprendida entre 0.2 y 0.6 mol/L 
			\item Localizar el momento en el que la temperatura crece a 0.1 K/min 
		\end{enumerate}
\end{enumerate}}

\qs{RCMP-E-2023}{En un reactor continuo mezcla perfecta de 2000 L alimentado por una corriente de 1200 L/min de 2 mol/L del compuesto A  a 300 K se lleva a cabo la reacción química:\\

A $\rightarrow$ B\\

Factor preexponencial = 1.5E12 min$^{-1}$\\
Energía de activación = 87.8 Kj/mol\\
Entalpía de reacción = -293 Kj/mol\\
Densidad = 0.9 Kg/L\\
Capacidad calorífica = 3.34 KJ/(Kg·K)\\

El reactor dispone de una camisa capaz de refrigerar a 2510 Kj/(K·min) que se situa a 370 K durante los primeros 60 min de operación. A continuación, la camisa se sitúa permanentemente a 380 K.


\begin{enumerate}[label=\bfseries\scriptsize\protect\circled{\footnotesize\Alph*}]
	\item Representar gráficamente los balances de materia y energía en estado estacionario y localizar los posibles estados estacionarios.
	\item Para unas condiciones iniciales iguales a la alimentación, obtener la evolución temporal de la concentración de A y de la temperatura y representar la trayectoria en la gráfica anterior.
\end{enumerate}}

\end{raggedright}
% \ex{Open Set and Close Set}{}
% \thm{}{If $x\in$ open set $V$ then $\exists$ $\delta>0$ such that $B_{\delta}(x)\subset V$}
% \begin{myproof}
% \end{myproof}

% \cor{}{By the result of the proof, we can then show...}
% \mlenma{}{Suppose $\vec{v_1}, \dots, \vec{v_n} \in \RR[n]$ is subspace of $\RR^n$.}
% \mprop{}{$1 + 1 = 2$.}

% \section{Random}
% \dfn{Normed Linear Space and Norm $\boldsymbol{\|\cdot\|}$}{Let $V$ be a vector space over $\bbR$ (or $\bbC$). A norm on $V$ is function $\|\cdot\|\ V\to \bbR_{\geq 0}$ satisfying \begin{enumerate}[label=\bfseries\tiny\protect\circled{\small\arabic*}]
% 		\item \label{n:1}$\|x\|=0 \iff x=0$ $\forall$ $x\in V$
% 		\item \label{n:2}	$\|\lambda x\|=|\lambda|\|x\|$ $\forall$ $\lambda\in\bbR$(or $\bbC$), $x\in V$
% 		\item \label{n:3} $\|x+y\| \leq \|x\|+\|y\|$ $\forall$ $x,y\in V$ (Triangle Inequality/Subadditivity)
% 	\end{enumerate}And $V$ is called a normed linear space.

% 	$\bullet $ Same definition works with $V$ a vector space over $\bbC$ (again $\|\cdot\|\to\bbR_{\geq 0}$) where \ref{n:2} becomes $\|\lambda x\|=|\lambda|\|x\|$ $\forall$ $\lambda\in\bbC$, $x\in V$, where for $\lambda=a+ib$, $|\lambda|=\sqrt{a^2+b^2}$ }


% \ex{$\bs{p-}$Norm}{\label{pnorm}$V={\bbR}^m$, $p\in\bbR_{\geq 0}$. Define for $x=(x_1,x_2,\cdots,x_m)\in\bbR^m$ $$\|x\|_p=\Big(|x_1|^p+|x_2|^p+\cdots+|x_m|^p\Big)^{\frac1p}$$(In school $p=2$)}
% \textbf{Special Case $\bs{p=1}$}: $\|x\|_1=|x_1|+|x_2|+\cdots+|x_m|$ is clearly a norm by usual triangle inequality. \par
% \textbf{Special Case $\bs{p\to\infty\ (\bbR^m$ with $\|\cdot\|_{\infty})}$}: $\|x\|_{\infty}=\max\{|x_1|,|x_2|,\cdots,|x_m|\}$\\
% For $m=1$ these $p-$norms are nothing but $|x|$.
% Now exercise
% \qs{}{\label{exs1}Prove that triangle inequality is true if $p\geq 1$ for $p-$norms. (What goes wrong for $p<1$ ?)}
% \sol{\textbf{For Property \ref{n:3} for norm-2}	\subsubsection*{\textbf{When field is $\bbR:$}} We have to show\begin{align*}
% 		         & \sum_i(x_i+y_i)^2\leq \left(\sqrt{\sum_ix_i^2} +\sqrt{\sum_iy_i^2}\right)^2                                       \\
% 		\implies & \sum_i (x_i^2+2x_iy_i+y_i^2)\leq \sum_ix_i^2+2\sqrt{\left[\sum_ix_i^2\right]\left[\sum_iy_i^2\right]}+\sum_iy_i^2 \\
% 		\implies & \left[\sum_ix_iy_i\right]^2\leq \left[\sum_ix_i^2\right]\left[\sum_iy_i^2\right]
% 	\end{align*}So in other words prove $\langle x,y\rangle^2 \leq \langle x,x\rangle\langle y,y\rangle$ where
% 	$$\langle x,y\rangle =\sum\limits_i x_iy_i$$

% 	\begin{note}
% 		\begin{itemize}
% 			\item $\|x\|^2=\langle x,x\rangle$
% 			\item $\langle x,y\rangle=\langle y,x\rangle$
% 			\item $\langle \cdot,\cdot\rangle$ is $\bbR-$linear in each slot i.e. \begin{align*}
% 				      \langle rx+x',y\rangle=r\langle x,y\rangle+\langle x',y\rangle	\text{ and similarly for second slot}
% 			      \end{align*}Here in $\langle x,y\rangle$ $x$ is in first slot and $y$ is in second slot.
% 		\end{itemize}
% 	\end{note}Now the statement is just the Cauchy-Schwartz Inequality. For proof $$\langle x,y\rangle^2\leq \langle x,x\rangle\langle y,y\rangle $$ expand everything of $\langle x-\lambda y,x-\lambda y\rangle$ which is going to give a quadratic equation in variable $\lambda $ \begin{align*}
% 		\langle x-\lambda y,x-\lambda y\rangle & =\langle x,x-\lambda y\rangle-\lambda\langle y,x-\lambda y\rangle                                       \\
% 		                                       & =\langle x ,x\rangle -\lambda\langle x,y\rangle -\lambda\langle y,x\rangle +\lambda^2\langle y,y\rangle \\
% 		                                       & =\langle x,x\rangle -2\lambda\langle x,y\rangle+\lambda^2\langle y,y\rangle
% 	\end{align*}Now unless $x=\lambda y$ we have $\langle x-\lambda y,x-\lambda y\rangle>0$ Hence the quadratic equation has no root therefore the discriminant is greater than zero.

% 	\subsubsection*{\textbf{When field is $\bbC:$}}Modify the definition by $$\langle x,y\rangle=\sum_i\overline{x_i}y_i$$Then we still have $\langle x,x\rangle\geq 0$}

% \section{Algorithms}
% \begin{algorithm}[H]
% \KwIn{This is some input}
% \KwOut{This is some output}
% \SetAlgoLined
% \SetNoFillComment
% \tcc{This is a comment}
% \vspace{3mm}
% some code here\;
% $x \leftarrow 0$\;
% $y \leftarrow 0$\;
% \uIf{$ x > 5$} {
%     x is greater than 5 \tcp*{This is also a comment}
% }
% \Else {
%     x is less than or equal to 5\;
% }
% \ForEach{y in 0..5} {
%     $y \leftarrow y + 1$\;
% }
% \For{$y$ in $0..5$} {
%     $y \leftarrow y - 1$\;
% }
% \While{$x > 5$} {
%     $x \leftarrow x - 1$\;
% }
% \Return Return something here\;
% \caption{what}
% \end{algorithm}
% \chapter{2}
\end{document}
